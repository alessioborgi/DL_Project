Given the MS Marco Dataset, we need to perform some preprocessing in order to create an association between queries and docids.
This Dataset is composed by 100k passages text which are partitioned into Training, Validation and Test. Each partition is composed by lists of passages texts and related queries.
We create a set $\mathcal{U} = \mathcal{O} \cup \mathcal{P}$, where $\mathcal{O}$ is the set composed by each segment (passage text) of the MS Marco Dataset, while $\mathcal{P}$ is the set of pseudo-queries from each segment of $\mathcal{O}$, which are artificially generated by a model (docT5query).

\subsection{Filtering with Rankers}
This new set $\mathcal{U}$ must be filtered in order to maintain only the relevant segments. 
This is done by using a ranker that takes as input each segment, which is used like a query and gives as output the first $\mathcal{k}$ ranked documents that have relevant information with respect to the segment.
This procedure is fundamental since checks if a segment has enough information to represent the document and can be used as a suitable query.
In our approach we have tried three different approaches:
\begin{itemize}
    \item \textbf{Lucene}: sparse ranker
    \item \textbf{Faiss}: dense ranker 
    \item \textbf{Hybrid}:
\end{itemize}
With this filtering procedure we are creating the real dataset $\mathcal{T}$, which will be the one used in the training phase. 

\subsection{Distillation Approach}
After the previous procedure, we may lose some information related to the choice of $\mathcal{k}$. In order to overcome this we want to explicitly model the relevance of various documents to a given query. This is performed by creating a Supervision Signal in the distillation task: $supsig_{t}^{dis} = (id_{t}, id_{1}, ..., id_{f})$.
This Supervision Signal is a string composed by the concatenation of the identifier of the segment $id_{t}$ and the $f$ result given by another Ranking procedure $id_{1}, ..., id_{f}$.
This process is expected to enable the model to acquire knowledge from the precise $id$ correpsondence with the text, and associate the pseudo-query to the list of most pertinent document identifier.