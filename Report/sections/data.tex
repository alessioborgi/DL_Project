In our work, we have used the first version of the MS Marco Dataset (reference bib), which is composed by 100k real Bing questions and human generated answers. The dataset is already partitioned into Training (82326 samples), Validation (10047 samples) and Test (9650 samples). Each partition is organized as a dictionary where the most important keys are the following:
\begin{itemize}
    \item \textbf{answers}: the answer related to the query based on the text informations.
    \item \textbf{passages}: contains another dictionary where we can find the complete corpus of the text and each passage that composes it.
    \item \textbf{query}: it is the question asked.
\end{itemize}
Since we are interested in the ranking of the documents, we have used the SimpleSearcher from pyserini \cite{pyserini}, in order to obtain the most relevant 1000 documents. 
The pre-processing applied to the dataset is explained in the following subsections.
\subsection{Tokenization}
Starting from the dataset, we have computed the maximum length of the inputs of the encoder (1797) and decoder (4), which will be useful during the tokenization process. We have used the pretrained tokenizer from the small version of the T5 model (reference). Our tokenized dataset is composed only by the following parts:
\begin{itemize}
    \item \textbf{Query}: tokenized version of the original query
    \item \textbf{Query and Corpus}: tokenized input text, which is composed by the concatenation of the query and the corpus
    \item \textbf{Document IDs}: tokenized version of the identificator of each document
    \item \textbf{Ranked Document IDs}: tokenized version of the ranked list of the first 1000 document ids
\end{itemize}
\subsection{DSI Multi-Generation}
In this subsection we have described the procedure to generate semantically structured document ids, which are characterized by the associations between queries and standard document ids. In other words, the docid should be able to capture some informations related to the semantics of the associated document. To do this, we have followed the algorithm provide by (reference ...)

\begin{algorithm}[H]
    \caption{Generating Semantically Structured Identifiers}
    \label{alg:semanticids}
    \begin{algorithmic}[1]
    \Require Document embeddings $X_{1:N}$, where $X_i \in \mathbb{R}^d$ generated by a small 8-layer BERT model with $c=100$
    \Ensure Corresponding docid strings $J_{1:N}$
    \Function{GENERATE\_SEMANTIC\_IDS}{$X_{1:N}$}
        \State $C_{1:10} \gets \Call{Cluster}{X_{1:N},\, k=10}$ \# k-means clustering
        \State $J \gets$ empty list
        \For{$i \gets 0$ to $9$}
            \State $J_{\text{current}} \gets [i] \times |C_{i+1}|$
            \If{$|C_{i+1}| > c$} \# recursion if there are more than c documents
                \State $J_{\text{rest}} \gets \Call{GENERATE\_SEMANTIC\_IDS}{C_{i+1}}$
            \Else
                \State $J_{\text{rest}} \gets [0,\dots,|C_{i+1}| - 1]$ \# assign arbitrary number from 0 to $c-1$
            \EndIf
            \State $J_{\text{cluster}} \gets \Call{elementwise\_Str\_Concat}{J_{\text{current}},\,J_{\text{rest}}}$
            \State $J \gets J.\Call{append\_elements}{J_{\text{cluster}}}$ \# Append all elements of $J_{\text{cluster}}$ to $J$
        \EndFor
        \State $J \gets \Call{reorder\_to\_Original}{J,\,X_{1:N},\,C_{1:10}}$
        \State \Return $J$
    \EndFunction
    \end{algorithmic}
    \end{algorithm}
After this procedure we have followed (reference understanding ...) to create the pseudo-queries, which are new queries generated by a model that takes in input the corpus of the documents.

\begin{align}
    \begin{cases}
        \mathcal{U} = \mathcal{O} \bigcup \mathcal{P} \\
            \mathcal{O} = \{ d_i^1, d_i^2, \cdots, d_i^m \} \\
            \mathcal{P} = \{ pq_i^1, pq_i^2, \cdots, pq_i^m \}
    \end{cases}
\end{align}
\subsection{Data Augmentation}

In this section, we present a first part of the introduced novel contributions that collectively enhance the dataset—through both semantic, stopwords and num2text augmentation as well as a more advanced POS-MLM augmentation. We decide to apply these techniques to the whole dataset corpus, by having Semantic, Stopwords and Num2Text Augmentation to be applied to all the corpuses, while having POS-MLM Augmentation to be applied to only 10\% of the corpuses' phrases.

\subsubsection{Semantic Augmentation: Stopwords and Num2Text}
Traditional DSI models commonly rely on a raw (or minimally processed) corpus, which often incorporates stopwords and punctuation as well as numbers. These low-value tokens can consume model capacity without contributing actually to semantic. In addition, purely numerical tokens are often not well captured in semantic embedding spaces. Furthermore, frequent words and repeated terms may overwhelm the more meaningful tokens, leading to reduced retrieval effectiveness and, at the same time, make the model process useless words, thereby extending training.

More in detail, the strategy we follow is:
\begin{itemize}
    \item \textbf{Stopword Removal:} By integrating established stopword and punctuation lists (using \texttt{nltk}) and frequency-based heuristics, we remove non-informative words (e.g., ``the,'' ``and,'' ``or''). Each token is compared against a set of known stopwords and punctuation. If the token is in this set, it is removed.
    \item \textbf{Num2Text Augmentation:} Numeric values (\textit{e.g.}, ``42'') are treated as discrete tokens and may lose contextual meaning if not converted into text. In practice, we transform (using the \texttt{inflect} library) numeric tokens into their textual representations (\textit{e.g.}, ``42'' $\to$ ``forty two'') so that the model can better capture semantic relationships involving numbers.
\end{itemize}

\subsubsection{POS-MLM Augmentation}
DSI models under examination often struggle to bridge the gap between document structure and actual query intent. Techniques like Part-of-Speech (POS) tagging have proven effective for syntactic parsing \cite{toutanova2003feature}, while Masked Language Modeling (MLM) has become a cornerstone of learning contextualized representations \cite{devlin2018bert}. Typically, these approaches are applied independently. Here we combine them in a novel manner.

\textbf{POS-MLM Augmentation} synthesizes syntactic insights from POS tagging with the context-driven capabilities of MLM, consisting in:
\begin{itemize}
    \item \textbf{POS Tagging:} Using spaCy \cite{honnibal2017spacy} (a lexical and syntactic parser), we assign part-of-speech labels (NOUN, VERB, ADJ, etc.) to each token. POS tags are used to identify tokens that have key syntactic roles, enabling a structured roadmap for masking.
    \item \textbf{Selective Token Masking:} Tokens holding key syntactic roles (here, verbs) are replaced with a placeholder token (e.g., \texttt{[MASK]}), so that the LM must focus on reconstructing them.
    \item \textbf{Predictive Training (MLM):} The masked sequences are processed by a pretrained Transformer model (e.g., \texttt{bert-base-uncased}), which infers the masked tokens. By leveraging the broader context, MLM reconstructs syntactic structures and promotes a deeper understanding of semantic relationships.
\end{itemize}

This dual focus enables more precise ranking of documents by their structural and semantic relevance, thus improving \textit{relevance ordering}, i.e., the arrangement of documents based on their contextual and structural pertinence.

