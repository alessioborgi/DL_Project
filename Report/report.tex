\documentclass{article}


\usepackage{PRIMEarxiv}

\usepackage[utf8]{inputenc} % allow utf-8 input
\usepackage[T1]{fontenc}    % use 8-bit T1 fonts
\usepackage{hyperref}       % hyperlinks
\usepackage{url}            % simple URL typesetting
\usepackage{booktabs}       % professional-quality tables
\usepackage{amsfonts}       % blackboard math symbols
\usepackage{nicefrac}       % compact symbols for 1/2, etc.
\usepackage{microtype}      % microtypography
\usepackage{lipsum}
\usepackage{subfiles}
\usepackage{fancyhdr}       % header
\usepackage{graphicx}       % graphics
\usepackage{algpseudocodex}
\usepackage{algorithm}
\usepackage{amsmath, amssymb}
\graphicspath{{media/}}     % organize your images and other figures under media/ folder

%Header
\pagestyle{fancy}
\thispagestyle{empty}
\rhead{ \textit{ }} 

% Update your Headers here
\fancyhead[LO]{Running Title for Header}
% \fancyhead[RE]{Firstauthor and Secondauthor} % Firstauthor et al. if more than 2 - must use \documentclass[twoside]{article}



  
%% Title
\title{Differentiable Search Indexing
%%%% Cite as
%%%% Update your official citation here when published 
\thanks{\textit{\underline{Citation}}: 
\textbf{Authors. Title. Pages.... DOI:000000/11111.}} 
}

\author{
  Alessio Borgi, Eugenio Bugli, Damiano Imola \\
  1952442, 1934824, 2109063 \\
  Sapienza Università di Roma \\
  Rome\\
  \texttt{\{borgi.1952442, bugli.1934824, imola.2109063\}@studenti.uniroma1.it} \\
}


\begin{document}
\maketitle


\begin{abstract}


This project presents a novel approach to enhancing the Differentiable Search Index (DSI), a neural inverted index framework, by introducing three data augmentation techniques: (1) converting numerical values to words (Num2Word), (2) removing stopwords, and (3) leveraging a Part of Speech Masked Language Modeling (POS-MLM) strategy. These augmentations aim to improve the robustness and effectiveness of the DSI model in diverse retrieval scenarios. Additionally, we propose and evaluate four advanced variants of the DSI model: DSI+LoRA, DSI+QLoRA, DSI+AdaLoRA, and DSI+ConvoLoRA, which integrate parameter-efficient fine-tuning methods to optimize performance and resource utilization.


\end{abstract}


% keywords can be removed
\keywords{DSI \and POS-MLM \and QLoRA \and Dynamic Pruning \and LoRA}

\section{Introduction} \subfile{sections/intro.tex}
\section{Dataset} \subfile{sections/data.tex}

\section{Model} \subfile{sections/model.tex}
\section{Training and Results} \subfile{sections/training.tex}

The documentation for \verb+natbib+ may be found at
\begin{center}
  \url{http://mirrors.ctan.org/macros/latex/contrib/natbib/natnotes.pdf}
\end{center}
Of note is the command \verb+\citet+, which produces citations
appropriate for use in inline text.  For example,
\begin{verbatim}
   \citet{hasselmo} investigated\dots
\end{verbatim}
produces
\begin{quote}
  Hasselmo, et al.\ (1995) investigated\dots
\end{quote}

\begin{center}
  \url{https://www.ctan.org/pkg/booktabs}
\end{center}


\subsection{Figures} 
See Figure \ref{fig:fig1}. Here is how you add footnotes. \footnote{Sample of the first footnote.}

\begin{figure}
  \centering
  \fbox{\rule[-.5cm]{4cm}{4cm} \rule[-.5cm]{4cm}{0cm}}
  \caption{Sample figure caption.}
  \label{fig:fig1}
\end{figure}

\subsection{Tables}
See awesome Table~\ref{tab:table}.

\begin{table}
 \caption{Sample table title}
  \centering
  \begin{tabular}{lll}
    \toprule
    \multicolumn{2}{c}{Part}                   \\
    \cmidrule(r){1-2}
    Name     & Description     & Size ($\mu$m) \\
    \midrule
    Dendrite & Input terminal  & $\sim$100     \\
    Axon     & Output terminal & $\sim$10      \\
    Soma     & Cell body       & up to $10^6$  \\
    \bottomrule
  \end{tabular}
  \label{tab:table}
\end{table}

\section{Conclusion}
Your conclusion here

\section*{Acknowledgments}
This was was supported in part by......

%Bibliography
\bibliographystyle{unsrt}  
\bibliography{references}  


\end{document}
